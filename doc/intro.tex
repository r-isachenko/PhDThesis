\addcontentsline{toc}{section}{Введение}
\chapter*{Введение}

Диссертационная работа посвящена построению математических моделей машинного обучения в пространствах высокой размерности.
Разработанные методы учитывают зависимости, имеющиеся в исходных данных, с целью построения простой и устойчивой модели.

\textbf{Актуальность темы.} 
В работе исследуется задача декодирования сигналов. 
Модель предсказывает отклик на входной исходный сигнал.
При построении модели возникает задача построения признакового пространства. 

Сложностью задачи является избыточность исходного описания данных. 
Исходное признаковое пространство является мультикоррелированным.
При высокой мультикорреляции финальная предсказательная модель оказывается неустойчивой.
Для построения простой, устойчивой модели применяются методы снижения размерности пространства~\cite{chun2010sparse,mehmood2012review}  и выбора признаков~\cite{katrutsa2015stress,li2017feature}.

В работе рассматриваются задачи с векторной целевой переменной. 
При предсказании векторной целевой переменной анализируется структура целевого пространства.
Предполагается, что целевое пространство имеет избыточную размерность.
В работе предложены методы, которые учитывают зависимости как в исходном пространстве объектов, так и в пространстве целевой переменной.

Методы снижения размерности пространства снижают размерность исходного пространства объектов, и, как следствие, сложность модели существенно снижается~\cite{tipping1999probabilisticpca,wold1975path,hotelling1992relations}. 
Алгоритмы снижения размерности находят оптимальные комбинации исходных признаков. 
Если число таких комбинаций существенно меньше, чем число исходных признаков, то полученное представление снижает размерность.
Цель снижения размерности -- получение наиболее репрезентативных и информативных комбинаций признаков для решения задачи.

Выбор признаков является частным случаем снижения размерности пространства~\cite{katrutsa2015stress,katrutsa2017comprehensive}. 
Найденные комбинации признаков являются подмножеством исходных признаков.
Таким образом отсеиваются шумовые неинформативные признаки.
Рассматриваются два типа методов выбора признаков~\cite{li2017feature,rodriguez2010quadratic,friedman2001elements}.
Первый тип методов не зависит от последующей предсказательной модели.
Признаки отбираются на основе из свойств исходных пространств, а не на основе свойств модели.
Второй тип методов отбирает признаки с учётом знания о предсказательной модели. 

После нахождения оптимального представления данных с помощью снижения размерности, ставится задача нахождения оптимальной метрики в скрытом пространстве объектов~\cite{davis2007information,kulis2012metric,yang2006distance,weinberger2009distance}.
В случае евклидова пространства естественным выбором метрики оказывается квадратичная норма.
Задача метрического обучения заключается в нахождении оптимальной метрики, связывающей объекты.

В диссертации решается задача декодирования с векторной целевой переменной. 
Для построения оптимальной модели декодирования временных рядов предлагаются методы проекции в скрытое пространство.
Исходные и целевые сигналы проецируются в пространство существенно меньшей размерности. 
Для связи проекцией исходного и целевого сигнала предлагаются методы согласования.
Рассматриваются многомодальные данные, природа каждой модальности различна.
Рассматриваются как линейные методы декодирования, так и их нелинейные обобщения.
Доказаны теоремы об оптимальности предложенных методов снижения размерности пространства, и методов выбора признаков.

\vspace{0.5cm}
\textbf{Цели работы.}
\begin{enumerate}
	\item Исследовать свойства решения задачи декодирования сигналов с многомерной целевой переменной.
	\item Предложить способ снижения размерности пространства, учитывающий зависимости как в исходном пространстве сигналов, так и в целевом пространстве.
	\item Предложить процедуру выбора признаков для задачи декодирования сигналов.
	\item Исследовать свойства линейных и нелинейных моделей для решения поставленной модели. Получить теоретические оценки оптимальности моделей.
	\item Провести вычислительный эксперимент для проверки адекватности предложенных методов.
\end{enumerate}


\vspace{0.5cm}
\textbf{Основные положения, выносимые на защиту.}
\begin{enumerate}
	\item Предложен метод снижения размерности пространства, отображающий независимую и целевую переменные в единое скрытое низкоразмерное представление.
	\item Предложены методы выбора признаков для задач с многомерной целевой переменной, учитывающие структуры пространств.
	\item Доказаны теоремы для симметричного и несимметричного учета значимости признаков, а также для минимаксной постановки задачи выбора признаков.
	\item Предложен алгоритм выбора наиболее релевантных параметров для оптимизации нелинейной модели. Исследованы свойства алгоритма.
	\item Предложен алгоритм метрического обучения для временных рядов с процедурой их выравнивания.
	\item Разработан программный комплекс, включающий прогностические модели данных высокой размерности. Проведены вычислительные эксперименты, подтверждающие адекватность моделей.
\end{enumerate}

\vspace{0.5cm}
\textbf{Методы исследования.}
Для достижения поставленных целей используются линейные и нелинейные алгоритмы регрессии.
Для анализа временных рядов используются классические авторегрессионные методы.
Для извлечения признаков используются частотные характеристики временного ряда.
Для построения скрытого пространства используются линейные методы снижения размерности пространства, их нелинейные модификации, а также нейросетевые методы.
Для выбора признаков наряду с классическими методами, используются методы, основанные на решении задачи квадратичного программирования.
Для построения метрического пространства используются методы условной выпуклой оптимизации.

\vspace{0.5cm}
\textbf{Научная новизна.}
Предложены методы построения моделей декодирования временных рядов, учитывающие структуры пространств исходных и целевых сигналов.
Предложены методы проекции сигналов в скрытое простраство, а также процедуры согласования образов.
Предложены алгоритмы выбора признаков с помощью квадратичного программирования.
Предложен метод выбора параметров для оптимизации с помощью выбора признаков.
Предложены алгоритмы построения оптимального метрического пространства для задачи анализа временных рядов.

\vspace{0.5cm}
\textbf{Теоретическая значимость.}
Доказаны теоремы об оптимальности предлагаемых моделей декодирования временных рядов.
Доказаны теоремы о корректности рассматриваемых моделей проекций в скрытое пространство.
Доказаны теоремы о достижении точки равновесия для предлагаемых алгоритмов выбора признаков. 

\vspace{0.5cm}
\textbf{Практическая значимость.}
Предложенные в работе методы предназначены для декодирования множества временных рядов сигналов электрокортикограмм, а также нестационарных временных рядов; выбора оптимальных частотных характеристик сигналов; выбора наиболее информативных параметров модели; классификации и кластеризации временных рядов физической активности.

\vspace{0.5cm}
\textbf{Степень достоверности и апробация работы.}
Достоверность результатов потверждена математическими доказательствами, экспериментальной проверкой результатов предалаемых алгоритмов на реальных данных, публикациями результатов в рецензируемых научных изданиях, в том числе рекомендованных ВАК. 
Результаты работы докладывались и обсуждались на следующих научных конференциях.
\begin{enumerate}
	\item Р. В. Исаченко. Метрическое обучение в задачах мультиклассовой класси- фикации временных рядов. \textit{Международная научная конференция <<Ломоносов>>}, 2016,~\cite{isachenko2016lomonosov}.
	\item R. G. Neychev, A. P. Motrenko, R. V. Isachenko, A. S. Inyakin, and V. V. Strijov. Multimodel forecasting multiscale time series in internet of things. \textit{Международная научная конференция  <<11th International Conference on Intelligent Data Processing: Theory and Applications>>}, 2016,~\cite{Neychev2016IDP}.
	\item Р. В. Исаченко, И. Н. Жариков, и А. М. Бочкарёв. Локальные модели для классификации объектов сложной структуры. \textit{Всероссийская научная конференция <<Математические методы распознавания образов>>}, 2017,~\cite{isachenko2017localmmro}.
	\item R. V. Isachenko and V. V. Strijov. Dimensionality reduction for multicorrelated signal decoding with projections to latent space. \textit{Международная научная конференция  <<12th International Conference on Intelligent Data Processing: Theory and Applications>>}, 2018,~\cite{Isachenko2018plsidp}.
	\item Р. В. Исаченко, В. В. Стрижов. Снижение размерности в задаче декодирования временных рядов. \textit{Международная научная конференция  <<13th International Conference on Intelligent Data Processing: Theory and Applications>>}, 2020,~\cite{Isachenko2020plsidp}.
\end{enumerate} 

Работа поддержана грантами Российского фонда фундаментальных исследований.
\begin{enumerate}
	\item {\color{red} ???}
\end{enumerate}

\vspace{0.5cm}
\textbf{Публикации по теме диссертации.}
Основные результаты по теме диссертации изложены в 6 печатных изданиях, {\color{red} ???} из которых изданы в журналах, рекомендованных ВАК.

\begin{enumerate}
	\item Исаченко Р. В., Катруца А. М. Метрическое обучение и снижение размерности пространства в задачах кластеризации // Машинное обучение и анализ данных, 2016. T. 2. № 1. С. 17--25~\cite{isachenko2016metricjmlda}.
	\item Исаченко Р. В., Стрижов В. В. Метрическое обучение в задачах мультиклассовой классификации временных рядов // Информатика и её применения, 2016. Т. 10. № 2. С. 48--57~\cite{isachenko2016metricia}.
	\item Isachenko R. et al. Feature Generation for Physical Activity Classification // Artificial Intelligence and Decision Making, 2018. № 3. С. 20--27~\cite{isachenko2018feature}.
	\item Isachenko R. V., Strijov V. V. Quadratic programming optimization with feature selection for nonlinear models // Lobachevskii Journal of Mathematics, 2018. Т. 39. № 9. С. 1179--1187~\cite{isachenko2018quadratic}.
	\item Isachenko R. V., Vladimirova M. R., Strijov V. V. Dimensionality Reduction for Time Series Decoding and Forecasting Problems //DEStech Transactions on Computer Science and Engineering, 2018. №. optim.~\cite{isachenko2018plsdestech}.
	\item Исаченко Р.В., Яушев Ф.Ю., Стрижов В.В. Модели согласования скрытого пространства в задаче прогнозирования // Системы и средства информатики, 2021. Т. 31 № 1.~\cite{isachenko2021concordance}
	\item {\color{red} expert systems}
\end{enumerate}

\vspace{0.5cm}
\textbf{Структура и объем работы.}

\vspace{0.5cm}
\textbf{Личный вклад.}
Все приведенные результаты, кроме отдельно оговоренных случаев, получены диссертантом лично при научном руководстве д.ф.-м.н. В. В. Стрижова.

\vspace{0.5cm}
\textbf{Краткое содержание работы по главам.}