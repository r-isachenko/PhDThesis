\addcontentsline{toc}{section}{Введение}
\chapter*{Введение}

Диссертационная работа посвящена построению математических моделей машинного обучения в пространствах высокой размерности.
Разработанные методы позволяют учесть зависимости, имеющиеся в исходных данных, с целью построения простой и устройчивой модели.


\textbf{Актуальность темы.} 

\vspace{0.5cm}
\textbf{Цели работы.}

\vspace{0.5cm}
\textbf{Задачи работы.}

\vspace{0.5cm}
\textbf{Основные положения, выносимые на защиту.}
\begin{enumerate}
	\item Метод снижения размерности пространства, отображающий независимую и целевую переменные в единое скрытое низкоразмерное представление.
	\item Методы выбора признаков для задач с многомерной целевой переменной, учитывающие структуры пространств.
	\item Алгоритм выбора наиболее влиятельных параметров для оптимизации нелинейной модели.
	\item Алгоритм метрического обучения для временных рядов с процедурой их выравнивания.
	\item Программный комплекс, включающий прогностические модели для высокоразмерных данных. Проведены вычислительные эксперименты, подтверждающие адекватность методов.
\end{enumerate}

\vspace{0.5cm}
\textbf{Методы исследования.}

\vspace{0.5cm}
\textbf{Научная новизна.}

\vspace{0.5cm}
\textbf{Теоретическая значимость.}

\vspace{0.5cm}
\textbf{Практическая значимость.}

\vspace{0.5cm}
\textbf{Степень достоверности и апробация работы.}

\vspace{0.5cm}
\textbf{Публикации по теме диссертации.}

\vspace{0.5cm}
\textbf{Структура и объем работы.}

\vspace{0.5cm}
\textbf{Личный вклад.}
Все приведенные результаты, кроме отдельно оговоренных случаев, по- лучены диссертантом лично при научном руководстве д.ф.-м.н. В. В. Стрижова.

\vspace{0.5cm}
\textbf{Краткое содержание работы по главам.}