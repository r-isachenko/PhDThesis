\addcontentsline{toc}{section}{Введение}
\chapter*{Введение}

Диссертационная работа посвящена построению математических моделей машинного обучения в пространствах высокой размерности.
Разработанные методы учитывают зависимости, имеющиеся в исходных данных, с целью построения простой и устойчивой модели.

\textbf{Актуальность темы.} 
В работе решается задача декодирования сигналов. 
Процесс декодирования заключается в восстановлении зависимости между двумя гетерогенными наборами данных.
Модель предсказывает отклик на входной исходный сигнал.
При построении модели возникает задача построения признакового пространства. 

Исследуется проблема избыточного исходного описания данных. 
Исходное признаковое пространство является мультикоррелированным.
При высокой мультикорреляции финальная прогностическая модель оказывается неустойчивой.
Для построения простой, устойчивой модели применяются методы снижения размерности пространства~\cite{motrenko2018multi,chun2010sparse,mehmood2012review}  и выбора признаков~\cite{katrutsa2017comprehensive,li2017feature}.

В работе рассматривается задача с векторной целевой переменной. 
Пространство целевых сигналов обладает избыточной размерностью. 
Методы снижения размерности, не учитывающие зависимости в целевом пространстве, являются не адекватными.
При предсказании векторной целевой переменной анализируется структура целевого пространства.
Предложены методы, которые учитывают зависимости как в пространстве исходных объектов, так и в пространстве целевой переменной.
Предлагается отобразить пространства исходных и целевых сигналов в скрытые подпространства меньшей размерности.
Для построения оптимальной модели предлагаются методы согласования скрытых пространств~\cite{wold1975path,rosipal2005overview,eliseyev2017recursive}.
Предложенные методы позволяют учесть регрессионную компоненту между исходным и целевым сигналами, а также авторегрессионную компоненту целевого сигнала.

Методы снижения размерности пространства понижают размерность исходного пространства объектов, и, как следствие, сложность модели существенно снижается~\cite{tipping1999probabilisticpca,wold1975path,hotelling1992relations}. 
Алгоритмы снижения размерности находят оптимальные комбинации исходных признаков. 
Если число таких комбинаций существенно меньше, чем число исходных признаков, то полученное представление снижает размерность.
Цель снижения размерности~--- получение наиболее репрезентативных и информативных комбинаций признаков для решения задачи.

Выбор признаков является частным случаем снижения размерности пространства~\cite{katrutsa2017comprehensive,katrutsa2015stress}. 
Найденные комбинации признаков являются подмножеством исходных признаков.
Таким образом отсеиваются шумовые неинформативные признаки.
Рассматриваются два типа методов выбора признаков~\cite{li2017feature,rodriguez2010quadratic,friedman2001elements}.
Первый тип методов не зависит от последующей прогностической модели.
Признаки отбираются на основе свойств исходных пространств, а не на основе свойств модели.
Второй тип методов отбирает признаки с учётом знания о прогностической модели. 

После нахождения оптимального представления данных с помощью снижения размерности, ставится задача нахождения оптимальной метрики в скрытом пространстве объектов~\cite{wang2017deep,davis2007information,kulis2012metric,yang2006distance,weinberger2009distance}.
В случае евклидова пространства естественным выбором метрики оказывается квадратичная норма.
Задача метрического обучения заключается в нахождении оптимальной метрики, связывающей объекты.

В качестве прикладной задачи анализируется задача построения нейрокомпьютерного интерфейса~\cite{wolpaw2000brain,allison2007brain}. 
Цель состоит в извлечении информации из сигналов мозговой активности~\cite{nagel2018modelling,zhang2020survey,chiarelli2018deep}. 
В качестве исходных сигналов выступают сигналы электроэнцефалограммы или электрокортикограммы. 
Целевым сигналом является траектория движения конечности индивидуума.
Задача модели построить адекватную и эффективную модель декодирования исходного сигнала в целевой сигнал.
Пространство частотных характеристик мозговых сигналов и авторегрессионное пространство целевых сигналов являются чрезвычайно избыточными~\cite{eliseyev2013recursive,eliseyev2011iterative}. 
Построение модели без учёта имеющихся зависимостей приводит к неустойчивости модели.

В диссертации решается задача декодирования с векторной целевой переменной. 
Для построения оптимальной модели декодирования сигналов предлагаются методы выбора согласованных моделей с проекцией в скрытое пространство.
Исходные и целевые сигналы проецируются в пространство существенно меньшей размерности. 
Для связи проекций исходного и целевого сигнала предлагаются методы согласования.
Рассматриваются гетерогенные наборы сигналов, природа источников измерений различны.
Рассматриваются как линейные методы декодирования, так и их нелинейные обобщения.
Доказаны теоремы об оптимальности предложенных методов выбора моделей.

\vspace{0.5cm}
\textbf{Цели работы.}
\begin{enumerate}
	\item Исследовать свойства решения задачи декодирования сигналов с векторной целевой переменной.
	\item Предложить методы снижения размерности пространства, учитывающие зависимости как в пространстве исходных сигналов, так и в целевом пространстве.
	\item Предложить процедуру выбора признаков для задачи декодирования сигналов.
	\item Исследовать свойства линейных и нелинейных моделей для решения поставленной модели. Получить теоретические оценки оптимальности моделей.
	\item Провести вычислительные эксперименты для проверки адекватности предложенных методов.
\end{enumerate}


\vspace{0.5cm}
\textbf{Основные положения, выносимые на защиту.}
\begin{enumerate}
	\item Исследована проблема снижения размерности сигналов в коррелированных пространствах высокой размерности. Предложены методы декодирования сигналов, учитывающие зависимости как в исходном, так и в целевом пространстве сигналов.
	\item Доказаны теоремы об оптимальности предлагаемых методов декодирования сигналов. Предлагаемые методы выбирают согласованные модели в случае избыточной размерности описания данных.
	\item Предложены методы выбора признаков, учитывающие зависимости как в исходном, так и в целевом пространстве. Предложенные методы доставляют устойчивые и адекватные решения в пространствах высокой размерности. 
	\item Предложены нелинейные методы согласования скрытых пространств для данных со сложноорганизованной целевой переменной. Предложен метод выбора наиболее релевантных параметров для оптимизации нелинейной модели. Исследованы свойства предлагаемого метода.
	\item Предложен алгоритм метрического обучения для временных рядов с процедурой их выравнивания.
	\item Предложен ряд моделей для прогнозирования гетерогенных наборов сигналов для задачи построения нейрокомпьютерных интерфейсов. Проведены вычислительные эксперименты, подтверждающие адекватность моделей.
\end{enumerate}

\vspace{0.5cm}
\textbf{Методы исследования.}
Для достижения поставленных целей используются линейные и нелинейные методы регрессионного анализа.
Для анализа временных рядов используются классические авторегрессионные методы.
Для извлечения признаков используются частотные характеристики временного ряда.
Для построения скрытого пространства используются линейные методы снижения размерности пространства, их нелинейные модификации, а также нейросетевые методы.
Для выбора признаков наряду с классическими методами, используются методы, основанные на решении задачи квадратичного программирования.
Для построения метрического пространства используются методы условной выпуклой оптимизации.

\vspace{0.5cm}
\textbf{Научная новизна.}
Предложены методы построения моделей декодирования сигналов, учитывающие структуры пространств исходных и целевых переменных.
Предложены методы проекции сигналов в скрытое пространство, а также процедуры согласования образов.
Предложены методы выбора признаков с помощью квадратичного программирования.
Предложен метод выбора параметров нелинейной модели для оптимизации с помощью выбора признаков.
Предложены методы построения оптимального метрического пространства для задачи анализа временных рядов.

\vspace{0.5cm}
\textbf{Теоретическая значимость.}
Доказаны теоремы об оптимальности предлагаемых моделей декодирования сигналов.
Доказаны теоремы о корректности рассматриваемых согласованных моделей проекций в скрытое пространство.
Доказаны теоремы о достижении точки равновесия для предлагаемых методов выбора признаков. 

\vspace{0.5cm}
\textbf{Практическая значимость.}
Предложенные в работе методы предназначены для декодирования множества временных рядов сигналов электрокортикограмм, а также нестационарных временных рядов; выбора оптимальных частотных характеристик сигналов; выбора наиболее информативных параметров модели; классификации и кластеризации временных рядов физической активности.

\vspace{0.5cm}
\textbf{Степень достоверности и апробация работы.}
Достоверность результатов подтверждена математическими доказательствами, экспериментальной проверкой результатов предлагаемых методов на реальных данных, публикациями результатов в рецензируемых научных изданиях, в том числе рекомендованных ВАК. 
Результаты работы докладывались и обсуждались на следующих научных конференциях.
\begin{enumerate}
	\item Р. В. Исаченко. Метрическое обучение в задачах мультиклассовой классификации временных рядов. \textit{Международная научная конференция <<Ломоносов>>}, 2016,~\cite{isachenko2016lomonosov}.
	\item R. G. Neychev, A. P. Motrenko, R. V. Isachenko, A. S. Inyakin, and V. V. Strijov. Multimodel forecasting multiscale time series in internet of things. \textit{Международная научная конференция  <<11th International Conference on Intelligent Data Processing: Theory and Applications>>}, 2016,~\cite{Neychev2016IDP}.
	\item Р. В. Исаченко, И. Н. Жариков, и А. М. Бочкарёв. Локальные модели для классификации объектов сложной структуры. \textit{Всероссийская научная конференция <<Математические методы распознавания образов>>}, 2017,~\cite{isachenko2017localmmro}.
	\item R. V. Isachenko and V. V. Strijov. Dimensionality reduction for multicorrelated signal decoding with projections to latent space. \textit{Международная научная конференция  <<12th International Conference on Intelligent Data Processing: Theory and Applications>>}, 2018,~\cite{Isachenko2018plsidp}.
	\item Р. В. Исаченко, В. В. Стрижов. Снижение размерности в задаче декодирования временных рядов. \textit{Международная научная конференция  <<13th International Conference on Intelligent Data Processing: Theory and Applications>>}, 2020,~\cite{Isachenko2020plsidp}.
\end{enumerate} 

Работа поддержана грантами Российского фонда фундаментальных исследований.
\begin{enumerate}
	\item 19-07-00885, Российский фонд фундаментальных исследований в рамках гранта <<Выбор моделей в задачах декодирования временных рядов высокой размерности>>.
	\item 16-37-00485, Российский фонд фундаментальных исследований в рамках гранта <<Развитие методов выбора признаков в условиях мультиколлинеарности>>.
	\item 16-07-01160, Российский фонд фундаментальных исследований в рамках гранта <<Развитие теории обучения по предпочтениям с использованием частично упорядоченных множеств экспертных оценок>>.
	\item 16-07-01154, Российский фонд фундаментальных исследований в рамках гранта <<Новые методы прогнозирования на базе субквадратичного анализа метрических конфигураций>>.
\end{enumerate}

\vspace{0.5cm}
\textbf{Публикации по теме диссертации.}
Основные результаты по теме диссертации изложены в 6 печатных изданиях, 5 из которых изданы в журналах, рекомендованных ВАК.

\begin{enumerate}
	\item Исаченко Р. В., Катруца А. М. Метрическое обучение и снижение размерности пространства в задачах кластеризации // Машинное обучение и анализ данных, 2016. T. 2. № 1. С. 17--25~\cite{isachenko2016metricjmlda}.
	\item Исаченко Р. В., Стрижов В. В. Метрическое обучение в задачах мультиклассовой классификации временных рядов // Информатика и её применения, 2016. Т. 10. № 2. С. 48--57~\cite{isachenko2016metricia}.
	\item Isachenko R. et al. Feature Generation for Physical Activity Classification // Artificial Intelligence and Decision Making, 2018. № 3. С. 20--27~\cite{isachenko2018feature}.
	\item Isachenko R. V., Strijov V. V. Quadratic programming optimization with feature selection for nonlinear models // Lobachevskii Journal of Mathematics, 2018. Т. 39. № 9. С. 1179--1187~\cite{isachenko2018quadratic}.
	\item Isachenko R. V., Vladimirova M. R., Strijov V. V. Dimensionality Reduction for Time Series Decoding and Forecasting Problems //DEStech Transactions on Computer Science and Engineering, 2018. №. optim~\cite{isachenko2018plsdestech}.
	\item Исаченко Р.В., Яушев Ф.Ю., Стрижов В.В. Модели согласования скрытого пространства в задаче прогнозирования // Системы и средства информатики, 2021. Т. 31 № 1~\cite{isachenko2021concordance}.
\end{enumerate}

\vspace{0.5cm}
\textbf{Структура и объем работы.}
Диссертация состоит из оглавления, введения, 6 глав, заключения, списка иллюстраций, списка таблиц, списка основных обозначений и списка литературы из 111 наименований. 
Основной текст занимает~\pageref{LastPage} страницы.

\vspace{0.5cm}
\textbf{Личный вклад.}
Все приведенные результаты, кроме отдельно оговоренных случаев, получены диссертантом лично при научном руководстве д.ф.-м.н. В. В. Стрижова.

\vspace{0.5cm}
\textbf{Краткое содержание работы по главам.}
В главе~\ref{ch:intro} вводятся основные понятия и обозначения. 
В разделе~\ref{sec:ch1:reg_model} формулируется задача восстановления регрессионной зависимости в пространствах высокой размерности.
В разделе~\ref{sec:ch1:decoding_task} ставится задача декодирования сигналов, приводится обзор методов анализа временных рядов.
В разделе~\ref{sec:ch1:dim_reduction} приводится обзор методов снижения размерности пространства для задачи декодирования сигналов.

Глава~\ref{ch:pls} посвящена проблеме построения согласованной модели декодирования.
В разделе~\ref{sec:ch2:concordance} вводятся понятия скрытого пространства и процесса согласования зависимостей, рассматриваются конкретные примеры методов снижения размерности пространства в терминах задачи согласования проекций.
В разделе~\ref{sec:ch2:pls_proof} приводится доказательство корректности работы линейных методов проекции в скрытое пространство.
Раздел~\ref{sec:ch2:superposition} посвящен рассмотрению случая аддитивной суперпозиции моделей декодирования, анализируются свойства моделей, входящих в суперпозицию.
Раздел~\ref{sec:ch2:exp_linear} содержит вычислительный эксперимент, демонстрирующий эффективность рассматриваемых линейных согласованных моделей декодирования сигналов.
В разделе~\ref{sec:ch2:exp_nonlinear} приводится вычислительный эксперимент для нелинейных модификаций согласованных моделей декодирования.

Глава~\ref{ch:qpfs} посвящена методам выбора признаков для задачи декодирования сигналов. 
Ставится задача выбора признаков как задача минимизации функции ошибки.
В разделе~\ref{sec:ch3:qpfs_feature_selection} рассматривается метод выбора признаков с помощью квадратичного программирования для случая скалярной целевой переменной.
Раздел~\ref{sec:ch3:mqpfs_feature_selection} посвящен обобщению скалярного случая на случай векторной целевой переменной.
Приводятся методы выбора признаков, позволяющие учесть зависимости в целевом пространстве.
Раздел~\ref{sec:ch3:exp_mqpfs} содержит вычислительный эксперимент, показывающий, что предложенные методы доставляют адекватные и устойчивые решения в сильно скоррелированных пространствах.

В главе~\ref{ch:newton_qpfs} рассматривается задача выбора активных параметров для оптимизации нелинейных моделей.
В разделе~\ref{sec:ch4:newton_qpfs_param_selection} ставится формальная задача выбора параметров модели как задача минимизации функции ошибки.
В разделе~\ref{sec:ch4:newton_algorithm} описан метод Ньютона для задачи нелинейной регрессии с квадратичной функцией потерь, а также для задачи логистической регрессии с кросс-энтропийной функцией потерь.
В разделе~\ref{sec:ch4:newton_qpfs_algorithm} приводится метод выбора параметров для рассматриваемых задач, использующий метод выбора признаков с помощью квадратичного программирования.
Раздел~\ref{sec:ch4:newton_qpfs_exp} содержит вычислительный эксперимент, доказывающий эффективность выбора параметров на множестве задач.

Глава~\ref{ch:metric_learning} посвящена построению оптимального метрического пространства для задачи анализа временных рядов.
Рассматриваются задачи кластеризации и классификации множества временных рядов активностей человека.
В разделе~\ref{sec:ch5:metric_learning_clustering} ставится задача поиска оптимальной метрики Махаланобиса для задачи кластеризации временных рядов.
В разделе~\ref{sec:ch5:metric_learning_adaptive} приводится алгоритм адаптивного метрического обучения для нахождения оптимального метрического пространства.
В разделе~\ref{sec:ch5:metric_learning_classification} рассматривается задача классификации временных рядов, использующая процедуру динамического выравнивания. 
Разделы~\ref{sec:ch5:exp_clustering} и \ref{sec:ch5:exp_classification} содержат вычислительные эксперименты на реальных временных рядах с акселерометра мобильного телефона.

Глава~\ref{ch:metamodels} посвящена методам построения оптимального признакового пространства для задачи анализа сигналов. 
В разделе~\ref{sec:ch6:feature_generation} ставится формальная задача порождения признакового описания.
Раздел~\ref{sec:ch6:feature_generation_models} содержит описание моделей порождения признакового пространства, основанных на экспертных знаниях и на порождающих моделях временных рядов.
В разделе~\ref{sec:ch6:feature_generation_classification} рассматривается задача классификации временных рядов по полученным признаковым описаниям.
В разделе~\ref{sec:ch6:exp_feature_generation} приводится вычислительный эксперимент, сравнивающий различные порождающие модели.