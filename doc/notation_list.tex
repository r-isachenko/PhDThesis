\addcontentsline{toc}{section}{Список основных обозначений}
\chapter*{Список основных обозначений}
$\bs = (s_1, \dots, s_T)$ -- временной ряд длины $T$ \\
$\cS = \{\bs_i\}_{i=1}^m$ -- множество из $m$ временных рядов \\
$\cS_t = \{s_{it}\}_{i=1}^m$ -- временное представление множества временных рядов \\
$\cS_{t, -h} = (\cS_{t - h}, \dots, \cS_{t})$ -- предыстория множества временных рядов длины $h$ в момент времени $t$ \\
$\cS_{t, p} = (\cS_{t + 1}, \dots, \cS_{t + p})$ -- горизонт прогнозирования множества временных рядов длины $p$ в момент времени $t$ \\
$\mathbf{g}$ -- порождающая функция \\
$\bX_{t, -h} = \mathbf{g}(\cS_{t, -h})$ -- признаковое представление предыстории \\
$\bY_{t, p} = \mathbf{g}(\cS_{t, p})$ -- признаковое представление горизонта прогнозирования \\
$f_{\text{AR}}: \bY_{t, -h} \rightarrow \bY_{t, p}$ -- авторегрессионная модель \\
$f_{\text{R}}: \bX_{t, -h} \rightarrow \bY_{t, p}$ -- регрессионная модель \\
$\bX = [\bx_1, \dots, \bx_m]^{\T} =  [\bchi_1, \dots, \bchi_n]$ -- матрица независимой переменной \\
$\bY = [\by_1, \dots, \by_m]^{\T} =  [\bnu_1, \dots, \bnu_r]$ -- матрица целевой переменной \\



