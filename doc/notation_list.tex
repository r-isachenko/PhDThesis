\addcontentsline{toc}{section}{Список основных обозначений}
\chapter*{Список основных обозначений}
\noindent 
$\bbX$~--- пространство исходной переменной \\
$\bbY$~--- пространство целевой переменной \\
$\bX = [\bx_1, \dots, \bx_m]^{\T} =  [\bchi_1, \dots, \bchi_n]$~--- исходная матрица \\
$\bY = [\by_1, \dots, \by_m]^{\T} =  [\bnu_1, \dots, \bnu_r]$~--- целевая матрица  \\
$\mathbf{f}^{\text{AR}}: \bbY \rightarrow \bbY$~--- авторегрессионная модель \\
$\mathbf{f}^{\text{R}}: \bbX \rightarrow \bbY$~--- регрессионная модель \\
$\bTheta$~--- параметры модели \\
$\cL(\bTheta, \bX, \bY)$~--- функция ошибки прогностической модели \\
$\bs = (s_1, \dots, s_T)$~--- временной ряд длины $T$ \\
$\cS = \{\bs^i\}_{i=1}^m$~--- множество из $m$ временных рядов \\
$\bx_t = ([\bs_{\bx}^1]_t, \dots, [\bs_{\bx}^m]_t)$~--- временное представление множества временных рядов \\
$\bX_{t,h} = [\bx_{t - h + 1}, \dots, \bx_{t}]^{\T}$~--- представление предыстории \\
$\bY_{t,r} = [\by_{t + 1}, \dots, \by_{t + p}]^{\T}$~--- представление горизонта прогнозирования \\
$\bbT$, $\bbU$~--- скрытые пространства для пространств исходной и целевой переменных \\
$\bphi_{\bx}: \bbX \rightarrow \bbT$, $\bphi_{\by}: \bbY \rightarrow \bbU$~--- функции кодирования \\
$\bpsi_{\bx}: \bbT \rightarrow \bbX$, $\bpsi_{\by}: \bbU \rightarrow \bbY$~--- функции декодирования \\
$\bh: \bbT\rightarrow \bbU$~--- функция связи \\
$g: \bbR^m \times \bbR^m \rightarrow \bbR$~--- функция согласования \\
$S(\ba', \bX, \bY)$~--- функция ошибки выбора признаков \\
$\bz$~--- вектор значимостей признаков \\
$\bQ_x$, $\bQ_y$~--- матрицы парных взаимодействий исходных признаков и целевых столбцов \\
$\bb$~--- вектор релевантностей признаков \\
$d_{\bA}: \bbX \times \bbX \rightarrow \bbR$~--- расстояние Махаланобиса c матрицей	трансформации~$\bA$ \\
$G\bigl(\mathbf{x}, \{\mathbf{c}_e\}_{e = 1} ^ K\bigr)$~---процедура выравнивания временных рядов относительно центроидов классов~$\{\mathbf{c}_e\}_{e = 1} ^ K$ \\
$\mathbf{g}: \bbR^T \rightarrow \bbX$~--- функция порождения признаков