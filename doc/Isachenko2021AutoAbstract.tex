\documentclass[11pt, a5paper]{dissert}

\usepackage{cmap}					% поиск в PDF
\usepackage{mathtext} 				% русские буквы в формулах
\usepackage[T2A]{fontenc}			% кодировка
\usepackage[utf8]{inputenc}			% кодировка исходного текста
\usepackage[english,russian]{babel}% локализация и переносы
\usepackage{indentfirst}
\usepackage{hyperref}
\frenchspacing

\usepackage{fullpage}
\usepackage{microtype}

\usepackage{lastpage}

\renewcommand{\epsilon}{\ensuremath{\varepsilon}}
\renewcommand{\phi}{\ensuremath{\varphi}}
\renewcommand{\kappa}{\ensuremath{\varkappa}}
\renewcommand{\le}{\ensuremath{\leqslant}}
\renewcommand{\leq}{\ensuremath{\leqslant}}
\renewcommand{\ge}{\ensuremath{\geqslant}}
\renewcommand{\geq}{\ensuremath{\geqslant}}
\renewcommand{\emptyset}{\varnothing}

%%% Дополнительная работа с математикой
\usepackage{amsmath,amsfonts,amssymb,amsthm,mathtools} % AMS
\usepackage{icomma} % "Умная" запятая: $0,2$ --- число, $0, 2$ --- перечисление

%% Номера формул
%\mathtoolsset{showonlyrefs=true} % Показывать номера только у тех формул, на которые есть \eqref{} в тексте.
%\usepackage{leqno} % Нумереация формул слева

%% Свои команды
\DeclareMathOperator{\sgn}{\mathop{sgn}}

%% Перенос знаков в формулах (по Львовскому)
\newcommand*{\hm}[1]{#1\nobreak\discretionary{}
	{\hbox{$\mathsurround=0pt #1$}}{}}

%%% Рисунки
\usepackage{tikz}
\usetikzlibrary{matrix}

%%% Работа с картинками
\usepackage{graphicx}  % Для вставки рисунков
\setlength\fboxsep{3pt} % Отступ рамки \fbox{} от рисунка
\setlength\fboxrule{1pt} % Толщина линий рамки \fbox{}
\usepackage{wrapfig} % Обтекание рисунков текстом
\usepackage{subfig}

%%% Работа с таблицами
\usepackage{array,tabularx,tabulary,booktabs} % Дополнительная работа с таблицами
\usepackage{longtable}  % Длинные таблицы
\usepackage{multirow} % Слияние строк в таблице
\usepackage{xcolor}

\usepackage{geometry}
\geometry{left=2.5cm}
\geometry{right=1.0cm}
\geometry{top=2.0cm}
\geometry{bottom=2.0cm}
\renewcommand{\baselinestretch}{1.0}

%https://tex.stackexchange.com/questions/163451/total-number-of-citations
\usepackage{totcount}
\newtotcounter{citnum} %From the package documentation
\def\oldbibitem{} \let\oldbibitem=\bibitem
\def\bibitem{\stepcounter{citnum}\oldbibitem}

\theoremstyle{definition}
\newtheorem{theorem}{Теорема}
\newtheorem{lemma}{Лемма}
\newtheorem{definition}{Определение}
\newtheorem{statement}{Утверждение}
\newtheorem{assumption}{Предположение}

\renewcommand{\contentsname}{Содержание}
\renewcommand{\contentsdesc}{Стр.}
\renewcommand{\chaptername}{Глава}
\renewcommand\thesection{\normalsize{\arabic{chapter}.\arabic{section}}}

\usepackage{algorithm}
\usepackage[noend]{algcompatible}
\def\algorithmicrequire{\textbf{Вход:}}
\def\algorithmicensure{\textbf{Выход:}}
\def\algorithmicif{\textbf{если}}
\def\algorithmicthen{\textbf{то}}
\def\algorithmicelse{\textbf{иначе}}
\def\algorithmicelsif{\textbf{иначе если}}
\def\algorithmicfor{\textbf{для}}
\def\algorithmicforall{\textbf{для всех}}
\def\algorithmicdo{}
\def\algorithmicand{\textbf{и}}
\def\algorithmicwhile{\textbf{пока}}
\def\algorithmicrepeat{\textbf{повторять}}
\def\algorithmicuntil{\textbf{пока}}
\def\algorithmicloop{\textbf{цикл}}
\def\algorithmiccomment#1{\quad// {\sl #1}}

\input{newcommands.tex}

\begin{document}

\begin{titlepage}
	\begin{flushright}
		{На правах рукописи}
	\end{flushright}
	\vspace{1.5cm}
	\begin{center}
		{Исаченко Роман Владимирович}
		\par
		\vspace{2cm}
		\textsc{Снижение размерности пространства в задачах декодирования сигналов}
		\par
		\vspace{2cm}
		{05.13.17~--- Теоретические основы информатики}
		\par
		\vspace{2cm}
		{АВТОРЕФЕРАТ\\
		диссертации на соискание ученой степени\\
		кандидата физико-математических наук}
	\end{center}
	\par
	\vspace{3.5cm}
	\begin{center}
		{Москва~--- 2021}
	\end{center}
\end{titlepage}


\setcounter{page}{2}

\noindent {Работа выполнена на Кафедре интеллектуальных систем Федерального государственного автономного образовательного учреждения высшего образования <<Московский физико-технический институт (национальный исследовательский институт)>>.

\vspace{0.1cm}

\vskip1ex\noindent
\begin{tabularx}{\linewidth}{@{}lX@{}}
  Научный руководитель: & \textbf{Стрижов Вадим Викторович}\\
  & доктор физико-математических наук, Федеральный исследовательский~центр <<Информатика и управление>> Российской академии наук, отдел интеллектуальных систем, ведущий научный сотрудник.
  \\[2pt]
  Официальные оппоненты: & \textbf{\color{red}Чуличков Алексей Иванович}\\
  & \textcolor{red}{доктор физико-математических наук, профессор, Федеральное государственное бюджетное образовательное учреждение высшего образования <<Московский государственный университет имени М. В.~Ломоносова>>, профессор кафедры математического моделирования и информатики физического факультета.}\\[2pt]
  & \textbf{\color{red}Зайцев Алексей Алексеевич}\\
  & \textcolor{red}{кандидат физико-математических наук, Автономная некоммерческая
образовательная организация высшего образования <<Сколковский институт науки и технологий>>, руководитель лаборатории в Центре по научным и инженерным вычислительным технологиям для задач с большими массивами данных.} \\[2pt]
  Ведущая организация: & \textcolor{red}{Федеральное государственное автономное образовательное учреждение высшего образования <<Санкт-Петербургский национальный исследовательский университет информационных технологий, механики и оптики>>.}
\end{tabularx}
\vskip2ex\noindent


\vspace{0.2cm}
\noindent Защита состоится{\color{red}~6~февраля 2020 года~в~13:00} на~заседании диссертационного совета Д 002.073.05 при Федеральном исследовательском центре <<Информатика и управление>> Российской академии наук (ФИЦ~ИУ~РАН) по адресу: 119333, г.\,Москва, ул.\,Вавилова, д.\,40.

\vspace{0.2cm}
\noindent С диссертацией можно ознакомиться в библиотеке Федерального государственного учреждения Федеральный исследовательский центр <<Информатика и управление>> Российской академии наук и на сайте http://www.frccsc.ru/

\vspace{0.2cm}
\noindent Автореферат разослан  \quad \quad \textcolor{red}{декабря 2021 года.}

\vspace{0.3cm}
\noindent И. о. ученого секретаря\\
диссертационного совета Д 002.073.05\\
д.т.н.
\hspace{12cm} И. А. Матвеев
}

\clearpage


%%%%%%%%%%%%%%%%%%%%%%%%%%%%%%%%%%%%%%%%%

\pretolerance=-1

\section*{Общая характеристика работы}
\label{ch:Intro}



\end{document}