\addcontentsline{toc}{section}{Заключение}
\chapter*{Заключение}

Основные результаты диссертационной работы заключаются в следующем.

В главе~\ref{ch:intro} введено понятие прогностической модели в пространствах высокой размерности.
Поставлена формальная задача декодирования сигналов.
Приведён обзор методов снижения размерности сигналов.
Рассмотрены линейные и нелинейные методы снижения размерности пространства.
Описаны методы, работающие с тензорными и многомодальными данными.

В главе~\ref{ch:pls} введено понятие скрытого пространства и процедуры согласования.
Рассмотрены методы построения согласованных моделей проекции в скрытое пространство.
Доказано утверждение об оптимальности линейных методов проекции в скрытое пространство.
Доказаны утверждения об оптимальности аддитивной суперпозиции моделей декодирования.

В главе~\ref{ch:qpfs} рассмотрена задача выбора признаков для декодирования сигналов.
Задача выбора признаков ставится как задача дискретной оптимизации. 
Приведён метод выбора признаков с помощью квадратичного программирования как решение релаксированной оптимизационной задачи.
Предложены обобщения процедуры выбора признаков для случая векторной целевой переменной: методы с симметричным и несимметричным учетом значимости целевых векторов, а также минимаксная постановка задачи.

В главе~\ref{ch:newton_qpfs} методы выбора признаков применяются к задаче выбора активных параметров при оптимизации нелинейных моделей.
Предложена модификация метода Ньютона для повышения стабильности процедуры оптимизации.
На каждом шаге алгоритма выбирается подмножество активных параметров для оптимизации алгоритмом выбора признаков с помощью квадратичного программирования.

В главе~\ref{ch:metric_learning} ставится задача выбора оптимальной метрики в пространстве временных рядов.
Приводится алгоритм кластеризации, использующий метрику Махаланобиса с обучаемой матрицей для вычисления расстояния между объектами. 
Для нахождения соответствия между временными рядами предложен алгоритм классификации с процедурой динамического выравнивания временных рядов.

В главе~\ref{ch:metamodels} рассмотрена задача порождения признакового пространства при решении задачи классификации временных рядов.
Признаковое пространство порождается с помощью метамоделей временных рядов.
В качестве метамоделей предлагаются авторегрессионная модель, метод анализа сингулярного спектра, аппроксимация сплайнами.