\addcontentsline{toc}{section}{Заключение}
\chapter*{Заключение}

Основные результаты диссертационной работы заключаются в следующем.

В~\autoref{ch:intro} рассматривается задача декодирования временных рядов.
Вводится понятие предсказательной модели.
Приведен обзор методов прогнозирования временных рядов.
Рассматриваются линейные и нелинейные методы снижения размерности пространства.
Описываются методы, работающие с тензорными и многомодальными данными.

В~\autoref{ch:pls} вводится понятие скрытого пространства и процедуры согласования.
Приведено доказательство корректности линейных методов проекции в скрытое пространство. {\color{red} Дописать.}

В~\autoref{ch:qpfs} рассматривается задача выбора признаков при декодирования сигналов.
Задача выбора признаков ставится как задача дискретной оптимизации. 
Алгоритм выбора признаков с помощью квадратичного программирования рассматривается как решение релаксированной оптимизационной задачи.
Приводятся обобщения процедуры выбора признаков для случая векторной целевой переменной.
Предлагаются алгоритмы с симметричным и несимметричным учетом значимости целевых переменных, а также минимаксная постановка задачи.

В~\autoref{ch:newton_qpfs} процедура выбора признаков применяется к выбору параметров для оптимизации нелинейных моделей.
Предлагается модификация метода Ньютона для увеличении стабильности процедуры оптимизации.
На каждом шаге алгоритма выбирается подмножество активных параметров для оптимизации алгоритмом выбора признаков с помощью квадратичного программирования.

В~\autoref{ch:metric_learning} ставится задача выбора оптимальной метрики в пространстве временных рядов.
Приводится алгоритм кластеризации, использующий метрику Махаланобиса с обучаемой матрицей для вычисления расстояния между объектами. 
Для нахождения соответствия между временными рядами приводится алгоритм классификации с процедурой динамического выравнивания временных рядов.

В~\autoref{ch:metamodels} рассматривается задача порождения признакового пространства при решении задачи классификации временных рядов.
Признаковое пространство порождается с помощью метамоделей временных рядов.
В качестве метамоделей используются авторегрессионная модель, метод анализа сингулярного спектра, аппроксимация сплайнами.